\documentclass[UTF8,a4paper]{ctexart}

% ==========Preamble==========

\usepackage{apacite}
\usepackage{fancyhdr}
\usepackage{geometry}
\usepackage[font=small,labelfont=bf,labelsep=quad,format=hang,textfont=it]{caption}
\usepackage{booktabs}
\usepackage{graphicx}
\usepackage{float}

\pagestyle{plain}
\CTEXsetup[format=\Large\bfseries]{section}
\bibliographystyle{apacite}

% ==========Title==========

\title{\bfseries 中国历史地理 } 
\author{谈昊\quad2020E8013282037}

% Your name in the first blank and your additional information in \thanks{}
\date{}
% delete \today if you don't want the date

% ==========Document==========

\begin{document}
\maketitle


% ==========Body==========

% Example article

\section{名词解释}
\subsection{郡县制}
郡县制下
土地是国家的, 官员代表中央实行管理 治理老百姓 。 可以说,
自秦以降,“普天之下,莫非王土”才真正实现。\par
中央政府将全国领土分为不同层次的行政管理区域,在各个区域内设置地方
政府,并授予地方政府一定行政、军事、财政、司法等权利。\par
郡县制的核心是集权政治,各 级地方政权只是中央王朝的一个环节,不具
备地方权力(只有“管理权 而无“所有权” )。郡县制实行以来可分为 三大阶
段,即郡政、州政、省政阶段 。
\begin{itemize}
    \item [(1)] 郡政阶段自秦至西汉,郡守集行政、经济、民政、司法权力为一身 。
    \item [(2)] 州政时期东汉末年开始(以州为中心管理天下)
    \item [(3)] 省政阶段:元明清,省级事权分属于不同官员,地方大吏督抚分设,
    互相牵制;军队独立于民政之外 。
\end{itemize}

\subsection{大运河}
大运河初指隋唐大运河,经元明清进一步改造后成为现在的模样。
\paragraph{隋唐大运河}
584年开凿广通渠,引渭水入黄河 。587年开凿山阳渎,联接江淮。605年开凿通济渠,自板诸引黄河水东经今开封、睢县、商丘、宿县、盱眙入淮。608年开凿永济渠,引沁水与淇水相接,循白沟故道北上,与永定河连接,抵达蓟县 。610
年重修京口(镇江)至余杭(杭州)段江南河。
\paragraph{元明清大运河}
1281 年开凿济州河,联通了济水与泗水。 1289 年开凿会通河,
将济州河向北延伸至临清。又继续向北至通州,形成联接南北的大运河。今北京积水潭即为通州渠的终点。元明清运河
同样连接原有水系但流向改变(裁弯取直) 。元朝运河 直接北上
抵达北京,不再经河南(政治中心北移与东移) 。运河在元朝时, 进入山东境需
要提水,越过高地进北京 。
\section{简述}
\subsection{中国古代行政区划界的两种原则}
氏族社会里只有部族居住地人口较集中,居住地周围是一片广大的狩猎区,
氏族间没有明确的边界概念。商、周时期依然表现为城邦国家的特征,商代都邑
之外为郊,为牧,为野;周代国外同样是郊,郊外是野。五服:甸、服、绥、要、
荒,但这只是个理想模式。
\par
春秋战国时期,战争多、人口多,疆界的功能越来越明显,有关边界划界
原则的记载也多起来。主要划界原则包括山川形便 和犬牙交错 原则。
\begin{enumerate}

    \item [(1)]
    随山川形便 

    山川形便指以天然山川作为行政区的边界,使行政区划与自然地理区划相一致。
    \paragraph{优点:}“广谷大川异制,民生其间异俗”。 独立的地理单元 有相似的环境 、民风、民俗等,便于管理;
    \paragraph{缺点:}完全以山川作为行政区边界,会成为一个完善的形胜之区,四塞之国。    易守难攻,割据者依山川之险与中央对抗。
    \item [(2)] 犬牙交错 

    行政区边界与山川形势有出入。这样的行政区使所有凭借山川之险的因素完
全消失。\par
军事上
、 经济上一些最重要的地方延伸出去 你中有我 我中有你 。
例如
隋朝江都郡 扬州 未以江为界 (扬州长江与大运河交汇,地理位置
重要); 再如 明南直隶:地跨长江与淮河,既有政治考虑,也有经济上的肥瘦搭
配(江南、江北、淮北) ;清 代 江苏省与安徽省 也 都跨江又跨淮 。


\end{enumerate}

\section{论述}
\begin{quote}
    \bfseries 根据生活经验或阅读体会,分析与历史地理相关的一种或若干种现象。
\end{quote}


我的家乡在山东济南。在济南市,“谈”姓人群应该只存在于我的家乡(具体到乡镇)附近。仿佛我们原本是不属于这里的,极有可能是从别的地方迁徙过来的。在向父辈询问之后,得到了原本的家乡是在山西附近这样的回答,还了解到“要问我的老家在哪里,山西洪洞大槐树”这样的俗语。
因此,顺着此条线索,经资料查询,发现历史上曾经有“洪洞大槐树移民”,这一大规模移民活动。
\par
朱元璋建立明朝后,为解决中原地区人口稀少问题,恢复中原社会生产,下令将山西人口迁徙到中原地区垦荒,这场移民潮被称为“洪洞大槐树移民”。

虽然称为“洪洞大槐树移民”,但不是将山西洪洞县的人口迁走,而是以此地作为集结中转的枢纽。

洪洞县位于晋南平原,农业发达,处于交通要道上,自古以来就设有驿站。当时洪洞县广济寺旁有一株高大的汉代古槐,贯通南北的驿道就从这棵古槐的树荫下通过。官方开始强制移民后,官府在广济寺设立办公点,将百姓聚集在古槐下,编排队伍,发放外迁证件和盘缠。移民上路后忍不住频频回头,直到古槐消失为止,使得洪洞的大槐树成为故乡留给移民的最后印象,也成为了移民后代关于故乡的记忆符号。


据《明实录》记载,明朝早期进行移民共18次,其中洪武年间10次,永乐年间8次,从山西移出上百万人,主要迁往豫冀皖等中原地区,到清末,移民后裔遍布全国,很多家族家谱清楚记录这段移民史。 

\clearpage


\end{document}
