\documentclass[UTF8,a4paper]{ctexart}

% ==========Preamble==========

\usepackage{apacite}
\usepackage{fancyhdr}
\usepackage{geometry}
\usepackage[font=small,labelfont=bf,labelsep=quad,format=hang,textfont=it]{caption}
\usepackage{booktabs}
\usepackage{graphicx}
\usepackage{float}

\pagestyle{plain}
\CTEXsetup[format=\Large\bfseries]{section}
\bibliographystyle{apacite}

% ==========Title==========

\title{\bfseries 计算神经科学综述 \\\begin{large}脑机接口\end{large}} 
\author{谈昊\quad2020E8013282037}

% Your name in the first blank and your additional information in \thanks{}
\date{\today}
% delete \today if you don't want the date

% ==========Document==========

\begin{document}
\maketitle

% ==========Abstract==========

\begin{center}
\parbox{130mm}{\zihao{-5}{\bfseries 摘\quad 要:}
% Abstract here
% An example is as follows
脑机接口(Brain–computer interface,BCI)是一种可以使人脑与机器产生交流的方法。它使得从大脑发出的信号可以用来控制外部的机器。目前,很多研究兴趣都集中在基于BCI的运动和认知疾病患者的神经康复上。几十年来,BCI已经成为运动和认知康复的一种替代治疗方法。以往的研究表明,BCI干预对恢复运动功能和受损大脑的恢复是有用的。基于脑电图(EEG)的BCI干预可以通过向受损大脑提供反馈,来揭示上肢恢复过程中神经可塑性的机制。BCI可以作为一种有用的工具,在肌萎缩侧索硬化症(ALS)等严重运动丧失病例中,可以帮助患者进行日常的交流和基本的运动。此外,最近的研究结果还报道了BCI对其他不同程度运动障碍疾病患者产生的的治疗效果,如痉挛性脑瘫、神经性疼痛等。除了运动功能恢复外,BCI还在改善注意力缺陷/多动症(ADHD)等认知疾病患者的行为方面发挥了作用。基于BCI的神经反馈训练主要是降低$\theta$ 节律和$\beta$节律的比值,或者使患者能够调节自己的皮质慢速电位,在提高注意力和警觉性方面都取得了进展。通过对多项证据确凿的临床研究的总结,本文介绍了BCI在运动和认知疾病(包括中风、ALS)中的临床应用的前沿成果。
\par
\vspace{1mm}
{\bfseries 关键词:}计算神经科学\quad BCI \quad EEG}
\end{center}

% ==========Body==========

% Example article

\section{前言}
脑机接口(BCI)为人类提供了一条大脑与外部环境之间的替代途径。近20年来,BCI相关研究异常活跃,该领域的文章层出不穷。这一新颖的技术因其多重应用而引起了科学家、临床医生和公众的广泛兴趣。目前,人们已经成功地将BCI应用于辅助交流、提供控制、恢复运动功能,甚至强化大脑能力。尤其是BCI的治疗应用是最令人振奋和最具前景的方向之一。BCI在中风、脊髓损伤(SCI)、肌萎缩性侧索硬化症(ALS)等运动疾病的康复方面具有巨大的潜力。
\par
对于那些严重瘫痪的患者,传统的治疗方法往往依赖于药物治疗,但是这些治疗方法无法恢复或增强神经通路。然而,BCI可能会成为一种很有前景的替代性神经康复方法。对于神经肌肉疾病患者来说,BCI控制的假肢装置可以弥补其运动能力的丧失,帮助患者进行简单的动作。此外,越来越多的证据表明,BCI的存在可以缓解神经功能障碍,恢复受损的感觉运动环路。经过治疗后,中风和急性SCI患者的功能和神经系统得到了极大的恢复。这鼓励了基于BCI的各项神经康复计划,其中许多计划确实能够改善患者的症状。
\par
BCI还可以作为神经发育问题儿童的辅助疗法,如自闭症和注意力缺陷/多动症(ADHD)。通过降低脑电图(EEG)中 $\Theta$节律和 $\beta$ 节律的比值(theta/beta ratio,TBR),基于BCI的神经反馈训练可以提高参与者的集中注意力。使患者能够控制缓慢的皮层电位的研究也能提高其认知能力。

\par
在本篇综述中,首先解释了BCI的原理,并简要介绍了BCI系统的前沿技术。然后,讨论了BCI在神经疾病康复中的普遍临床应用,包括运动损伤和认知灾难。我们还阐明了神经康复治疗的一些基本机制。最后,我们讨论了现有研究中一些普遍存在的缺陷和未来研究的方向。

\section{发展}
脑机接口(BCI)最初的定义是 "个体能够向外界环境发出信息或指令,但是不经过大脑的周围神经和肌肉的正常输出途径的通信系统"。例如,在基于脑电图(EEG)的BCI中,信息可以通过特定的EEG特征直接解码。
\par
重度运动瘫痪患者需要不依靠于肌肉的通信技术。这种通信对这些病人的生活质量至关重要(Bach,1993年)。特别是当患者患有神经系统或肌肉疾病时,如肌萎缩侧索硬化症(ALS),随着疾病的发展,会导致运动障碍。2012年,\cite{wolpaw2002brain}拓宽了脑机接口的含义,将其定义为 "一种测量中枢神经系统(CNS)活动并将其转化为人工输出的系统,以替代、恢复、增强、补充或改善中枢神经系统的自然输出,从而改变中枢神经系统与其外部或内部环境之间正在进行的相互作用",提出了将该技术用于dierent应用的可能性。用户自愿调节他/她的大脑活动,以产生对周围环境的特定指令来替换或恢复受损的肌肉能力\cite{abiri2019comprehensive,aloise2013asynchronous,pichiorri2015brain}。特别是,针对健康用户的BCI可以用来增强人与周围环境的互动。在这方面,BCI(即被动BCI,pBCI,\cite{di2018eeg,blankertz2016berlin,cartocci2015mental,zander2011towards,valeriani2019brain,vecchiato2016investigation,astolfi2012cortical,sciaraffa2017brain})能够获得不以自愿控制为目的而产生的任意大脑活动(即用户状态的隐含信息),例如,工作量、注意力、情绪以及大多数一般任务引起的状态。这些状态只能用传统的方法如主观(如问卷)或行为(如反应时间)测量来检测,可靠性较弱\cite{zander2011context}。基于pBCIs的系统可以直接在一个闭环中使用这些关于用户状态的信息来自动修改用户正在与之交互的界面的行为(即自适应自动化),或者只是实时地通知用户自己或其他人的危险行为(例如,超载\cite{borghini2017new},或注意力不集中\cite{di2019brain,sebastiani2020neurophysiological}),这些行为可能会增加人为错误的概率,从而诱发可能的不安全情况。
\par
近几年来,BCI领域从多个角度实现了几次巨大的飞跃。例如,在前端图形用户界面(GUI)方面已经产生了许多作品,最近发表在《脑科学》杂志上的评论文章 "Brain-Computer Interface Spellers: A Review "表示,"多年来,科学家们一直致力于拼写系统的研究,使其更快、更准确、更方便用户使用,最重要的是,能够与传统的交流方式竞争"\cite{rezeika2018brain}。在这方面,在BCI系统下运行的后端算法(即分类技术)方面也做出了巨大的努力\cite{schettini2014self},通过使用越来越少的特征(即脑电图传感器),却可以实现高识别精度(例如,有目标与无目标,低工作量与高工作量),同时实现高信息传输率(ITRs)。在这方面,基于生理数据分析的机器学习和深度学习方法在过去十年中经历了快速发展,因为这种方法能够提供解码和表征任务相关的大脑状态(即从多维度问题减少到一维度问题),并将它们与非信息性大脑信号区分开来(即提高信噪比)。在这方面,\cite{arico2017passive,arico2018passive}通过测试BCI系统在日常生活中的应用,证明了这种技术的成熟性和有效性。
\par
最后,技术上的增强与脑电图记录耳机有关,它最终可以让BCI系统进入市场,尤其是日常生活应用。近年来,许多公司一直致力于开发更多可穿戴和微创的生物信号采集设备。特别是在脑电图系统方面,目前正在开发干式传感器(即不需要任何导电凝胶),或者最终使用水基技术代替经典的凝胶技术,从而实现高信号质量和更高的舒适度(如\cite{von2017headgear})。现在普遍认为,凝胶型电极仍然要被认为是黄金标准\cite{tallgren2005evaluation,lopez2014dry},然而,湿电极和干电极之间的差距正在越来越小\cite{di2019dry}。关于这些创新的干式脑电图电极的比较和验证,文献中已经有一些尝试。在这方面,最近Di Flumeri及其同事\cite{di2019dry}发表了一篇论文,旨在通过比较三种不同类型的干式电极与传统电极(即基于凝胶的电极),来评估EEG干式电极行业所达到的成熟度。这项工作的结果突出了干式脑电图解决方案所达到的高质量水平,因为所有测试的电极都能够保证与湿式电极相同的质量水平,同时可以大大减少蒙太奇的时间,提高用户的舒适度。
\section{研究方法和内容}
一个标准的BCI系统包括五个功能模块,即脑信号采集、信号预处理、特征提取、特征分类和输出设备。利用这些模块,BCI系统可以对用户的脑信号进行解码,并将其转化为计算机指令,用于控制外部设备或神经康复系统。
\par
信号采集模块对用户的脑信号进行记录、放大和数字化,例如,可以是与运动执行/想象密切相关的指定脑电图节律。可以通过两种主要方法从中枢神经系统(CNS)记录脑信号:侵入式和非侵入式方法。除了运动诱导的脑信号外,视觉和/或肌肉诱导的信号也已被研究\cite{SHIH2012268}。
\par
一般来说,记录的脑信号(如EEG信号)会受到伪影或噪声的污染,如肌电图(EMG)、心电图(EOG)、心电图(ECG)伪影和电力线噪声等。信号预处理的主要目的是提高记录的脑信号的信噪比(SNR)。
\par
根据伪影的特点,通常采用时间滤波、频谱滤波和空间滤波技术分别去除时间域、频谱域和空间域的伪影或噪声。例如,电力线噪声可以通过频谱滤波来去除。EOG伪影是由眨眼和眼球运动引起的,一般会在低频区域引起高幅度的模式。空间滤波,将多通道EEG信号线性组合以提高SNR,可以使用数据独立或数据驱动的方法进行。数据独立的方法采用共同的平均参考和Laplacian滤波器,数据驱动的方法采用共同的空间模式、规范相关分析、独立成分分析或主成分分析\cite{lotte2018review,bashashati2007survey}。经过信号预处理后,可以提高记录脑信号的SNR。
\par
特征提取是用来从大脑信号中提取特征,而这些特征是代表意图的。两种广泛使用的特征是频段功率特征和时间特征。频段功率特征是指脑电信号在指定频段(或节奏)内的功率,如$\alpha$ 、$\beta$ 、$\mu$ 和$\gamma$ 频段。例如,在
感知运动皮层上的$\frac{\mu}{\beta}$节律是基于运动图像的BCI中的一个重要特征[19,20]。时间特征在基于P300(或事件相关电位)的BCI中起着重要作用,在BCI中,EEG电位的变化与给定事件或刺激有时间上的联系\cite{hochberg2012reach}。
\par
特征分类是应用特征提取模块提取的脑电图特征进行识别。在过去的几十年里,人们开发了各种算法来对脑电图相关特征进行分类\cite{lotte2018review,bashashati2007survey}。常用的分类器基于线性方法,如线性判别分析和支持向量机,因为当标记数据和BCI任务知识有限时,线性方法可以进行相对稳定的计算\cite{muller2003linear}。最近,深度神经网络也被应用于BCI应用中脑电图特征的分类\cite{lotte2018review}。
\par
通过输出设备触发动作,让用户在现实世界中行动,而不考虑周围神经和肌肉的参与。执行器包括功能性电刺激(FES)、神经刺激、假肢装置和外骨骼。
\par
FES通过电刺激支配肌肉,还能使本体感觉反馈到大脑。机器人假肢,如阿凡达臂,对运动障碍急性期的患者效果较好,可启动被动运动。近年来,由FES和主动矫形器部件组成的混合FES系统作为恢复日常操作能力的有效途径得到了更广泛的应用\cite{rupp2012development}。
\par
运动除了是预期的结果外,还能向大脑提供反馈,进而修改运动执行过程,提高速度和准确性\cite{wolpaw2002brain,mcfarland2017therapeutic}。在最理想的情况下,大脑还会自行修改信号特征,形成双向适应\cite{wolpaw2002brain}。
\section{技术领域}
\subsection{BCI在中风康复中的应用}
中风作为第二大常见死因,全世界每年约有1500万患者\cite{members2017heart,bene2012interface},而药物治疗是减轻其症状的传统治疗方法。令人鼓舞的是,最近的研究表明,BCI是一种很有前途的技术,可以使慢性中风的患者的运动能力得到康复\cite{song2014characterizing,song2015dti,muralidharan2011extracting}。中风后的BCI干预既可用于BCI辅助康复,也可作为干预的决策指导\cite{doi:10.1080/2326263X.2013.876724}。中风治疗中应用BCI的目的是恢复运动和认知能力,增强康复过程中的神经可塑性\cite{mcfarland2017therapeutic}。
\par
BCI治疗的临床疗效已在一组慢性脑中风患者中通过Fugl-Meyer运动评估显示\cite{mrachacz2016efficient},证明BCI干预后急性脑中风患者的神经生理和行为有明显改善。BCI相关的靶向神经反馈(NF)对大脑结构和功能的影响,可通过磁共振成像(MRI)快速检测。BCI诱导的空间特异性脑可塑性有望针对脑中风后的功能缺陷进行治疗干预\cite{nierhaus2019immediate}。
\par
基于EEG的运动影像(MI)BCI系统也被发现可以恢复运动功能。MI可以在没有真实运动的情况下通过运动尝试唤起,因此可以作为重症中风患者传统物理康复的辅助手段\cite{crosbie2004adjunctive}。在运动任务过程中,脑电图记录的$\alpha$ 和$\beta$节律呈现事件相关的去同步化/同步化(ERD/ERS)趋势,这为脑中风康复中神经可塑性的机制提供了深入的认识\cite{carino2019longitudinal}。BCI康复的个人设计方案是大脑区域和功能之间关于行为、临床和神经生理变化的精确耦合。
\par
运动康复过程中,还可以通过本体感觉与肌肉运动或动作观察向患者大脑提供反馈(图1)。脑中风患者的成功康复需要大脑活动反馈与其最初的运动意图相配合,从而形成一个闭环系统。BCI辅助运动通过提供大脑活动和身体对CNS的反应的宝贵信息,增强了闭环回路。这有利于中枢神经系统的可塑性,并导致正常大脑功能的恢复或功能控制权转移到未受损的大脑区域\cite{doi:10.1080/2326263X.2013.876724}。
\par
虽然本体感觉反馈在恢复神经可塑性方面的积极功能已经得到验证,但最近Vourvopoulos等人进行的一项研究结合了虚拟现实(VR)和BCI的原理,作为运动反馈的组合\cite{vourvopoulos2019effects}。作者测量了脑电图和肌电图作为运动意图的信号,以驱动指令并启动虚拟呈现的虚拟人手臂的运动,同时可见VR的观察和被动运动过程也为患者的大脑想象提供反馈。该研究提出了一种有用的NF VR-BCI范式,用于脑中风后的高效运动康复,并证明了其作为没有本体感觉反馈途径的个体的替代工具的可行性。此外,运动障碍较严重的患者通过基于EEG的BCI实现了较好的康复,而主动运动的患者则更多地受益于多模态平台的EMG反馈。这一发现可以为今后对不同阶段脑中风患者的信号选择提供参考,以指示更精确的大脑活动信息。此外,脑连通性指数的评估研究也证明了BCI干预在脑中风后认知康复中的治疗可能性\cite{toppi2014time}。

\subsection{运动功能严重丧失的情况下的BCI}
对于完全被锁住的患者,BCI允许患者向外部世界传达信息和指令。初步研究表明,BCI是帮助ALS患者进行日常交流和基本运动的有效工具\cite{doi:10.1080/2326263X.2013.876724,kim2016clinical,wolpaw2018independent,carelli2017brain}。一项对42名ALS患者的随访研究也证明了ALS患者独立在家使用BCI的可行性,为护理人员和患者提供了缓解\cite{wolpaw2018independent}。
\par 
此外,最近的研究结果还报道了BCI-FES对痉挛性脑瘫儿童的治疗效果。这凸显了BCI在临床上引发失神经肌肉收缩、恢复瘫痪运动功能、治疗肌无力的可能性\cite{kim2016clinical}。
\par
在锁定的患者中使用BCI系统有两大局限性。首先,BCI准确性不足,尤其是在拼写方面,意味着患者的基本交流需求仍未得到满足。目前正在进行研究,以提高基于BCI的交流速度和准确性\cite{santhanam2006high}。此外,长期使用BCI设备可能会导致视力受损等副作用。克服这一限制的方法之一是使用听觉刺激而非视觉刺激\cite{wolpaw2018independent}。

\section{预测和展望}
本文着重介绍了脑机接口的相关概念以及研究方法和相关的发展过程。并且介绍了一些脑机接口在疾病治疗方向上的应用。与传统的药物治疗相比,BCI是一种很有前途的技术,可以恢复其神经系统的功能来使慢性中风的患者的运动能力得到康复。
\par 
但是,目前来看以后脑机接口的作用远远不止于此。脑机接口(BCI)将大脑的神经元活动转化为驱动外部效应器或影响内部身体部位和功能的信号。人类与设备互动的想法是BCI的支柱。在未来几年内,我们也许可以只用大脑来修改我们的PowerPoint演示文稿或Excel文件。目前一些原型机可以将大脑活动翻译成文字或计算机的指令,理论上,随着技术的进步,我们将看到人们在工作中使用BCI来写备忘录或报告。我们也可以想象一个能自动适应压力水平的工作环境。BCI可以检测工作者的精神状态,并相应调整附近的设备(智能家居利用)。
\par
但是因此带来的就是隐私问题。这种监控注意力水平的能力为管理者创造了新的可能性。例如,公司可以访问特定的 "BCI人力资源仪表盘",其中将实时显示所有员工的大脑数据。在每个年度绩效评估结束时,是否也会通过BCI来分析和比较工作水平?大脑信息可能会使公司能够观测雇员的注意力有多集中,并允许他们相应地调整员工的工作量。同样,也有很多被滥用的可能性。
\par
不过,这项技术还是在慢慢进入大众市场。越来越多的初创企业和大型科技公司正在研发更安全、更准确、更便宜的BCI。期望BCI在蓬勃发展的同时,尽快开始构建相应的法律和监管体系,以应对潜在的风险和收益。

% ========About apacite===========
% \cite{baddeley1992working}  => (Baddeley, 1992)
% \citeA{baddeley1992working} => Baddeley (1992)

% ==========References==========
\renewcommand{\refname}{参考文献}
% enter your .bib file name in the parentheses in the following line.
\bibliography{ref}

\clearpage


\end{document}
