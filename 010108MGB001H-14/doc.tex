\documentclass[UTF8,a4paper]{ctexart}

% ==========Preamble==========

\usepackage{apacite}
\usepackage{fancyhdr}
\usepackage{geometry}
\usepackage[font=small,labelfont=bf,labelsep=quad,format=hang,textfont=it]{caption}
\usepackage{booktabs}
\usepackage{graphicx}
\usepackage{float}

\pagestyle{plain}
\CTEXsetup[format=\Large\bfseries]{section}
\bibliographystyle{apacite}

% ==========Title==========

\title{\bfseries 自然辩证法在建筑、教育、企业管理中的应用 } 
\author{谈昊\quad2020E8013282037\quad217}

% Your name in the first blank and your additional information in \thanks{}
\date{\today}
% delete \today if you don't want the date

% ==========Document==========

\begin{document}
\maketitle

% ==========Abstract==========

\begin{center}
\parbox{130mm}{\zihao{-5}{\bfseries 摘\quad 要:}
% Abstract here
% An example is as follows
自然辩证法是研究自然界和科学技术发展的一般规律以及人类认识和改造自然的一般方法的科学,基于系统观、辩证观、科学观对管理实践的理论深化,在社会发展中发挥了重要的作用。本文详细阐述了其在建筑、教育、企业管理方面的重要作用。
\par
\vspace{1mm}
{\bfseries 关键词:}自然辩证法\quad 建筑 \quad 教育 \quad 企业管理}
\end{center}

% ==========Body==========

% Example article

\section{前言}
自然辩证法是关于自然界和自然科学发展的普遍规律的科学,是马克思主义哲学的分支学科,是由马克思和恩格斯共同创立的,恩格斯的枟自然辩证法枠这部著作为这门学科的诞生奠定了基础,开辟了一个新的研究领域。自然辩证法的创立不是来自于思辩和猜测,而是在19世纪自然科学发展、哲学发展和社会发展的科学文化背景下,经过长期潜心研究而建立的一个完整的、相对独立的科学理论知识体系,是人类科学认识发展的必然结果。自然辩证法的内容是阐释马克思主义的自然哲学、科学哲学和技术哲学,是关于自然界和科学技术发展的一般规律及人类认识和改造自然的一般方法的科学,是系统化、理论化的自然观、科学技术观和科学技术研究的方法论。自然辩证法是一个开放性的科学理论知识体系,随着科学技术的不断发展,自然辩证法的内容也不断得到证实、丰富和发展。\cite{RN14}
\section{自然辩证法在建筑中的应用}
近年来,由于科学技术在现代化的大环境下得到前所未有的快速发展,人类认识自然、改造自然的能力也逐渐得到提高,建筑设计要遵循自然辩证法原理得到大力支持,同时随着科学技术的日新月异,人类能够通过更多途径来实现生态建筑设计,优化建筑设计和人居环境的质量。自然辩证法的研究对象是以人与自然的总体关系为核心,研究目的在于如何合理地解决人与自然之间的矛盾,建筑是人类生存、社会发展的空间载体,人与自然、社会的发展需要都是建筑师重点关注的对象,在设计中遵循自然辩证法成为其基本的设计理念。自工业革命以来,科学技术推动生产方式不断进步,社会经济不断增长,生活水平不断提高,消费者的购买力越来越强,随之而来的是自然资源的迅速流失,人类一味索取导致了环境的恶化、资源的紧缺,生态危机开始来临,人类对自然保护的意识逐渐增强,开始积极探索人与自然和谐共处的发展道路。随着城市化的不断发展,建筑业市场前景广阔,同时,环境保护和可持续发展的理论也显得格外重要,生态建筑设计逐渐兴起,人们渴望获得良好的生活环境,渴望与自然相伴,与青山绿水相邻。
\par
\cite{RN11}提出应尽量利用一切可以利用的建筑物和建筑材料,以有助于大大减少资源的消耗,降低消耗量,同时也可以减少建筑垃圾的产生,以达到对自然生态的保护,在建筑的设计过程中,重视旧有建筑的维护更新和对原有建筑的重复建设;其次,充分考虑原有的自然生态条件的利用率,注重协调建筑和自然生态环境的关系,以防施工时发生水土流失和其他自然灾害,并保持水土,调节气候,降低污染;设计中应当结合地理位置、方向等,采用自然通风、天然采光的方法,降低建筑对能源的依赖。

\section{自然辩证法在教育中的应用}
“自然辩证法概论”是一门自然科学、技术科学、社会科学与思维科学相交叉的学科,在内容上涉及自然科学、技术科学、社会科学和思维科学,具有明显的综合学科的性质。自然辩证法课程的概括性、综合性、哲理性和交叉学科的特点,使它在我国目前高等学校的课程设置中有着重要的地位和作用。它弥补了我国高等学校课程设置中缺乏整体性课程、缺乏交叉学科的不足,对于扩大学生的知识面,提高学生的整体思维能力有着重要的作用。
\par
\cite{RN13}提出,与具体的各门自然科学相比,由于自然辩证法具有世界观和方法论的高度,使它有突出的德育功能。自然辩证法又有把德育教育窝于智育教育之中的特点。自然辩证法课程对学生的思想教育是渗透到科学、技术专业之中的。它通过学生对自然界和自然科学的辩证发展规律的认识,以及对自然科学一般研究方法的理解和掌握这样一些智育的培养中,使学生逐步树立起对自然界和自然科学的正确观点。

\section{自然辩证法在企业管理中的应用}
企业管理,是协调人与自然关系的实践活动,必然要遵循自然辨证法。系统观、辨证观和科学观组成了自然辨证法理论体系,指导着企业管理的实践。
\par
\cite{RN12}提出从系统观赏来讲,管理者把企业看成一个系统,针对企业管理工作中经常遇到的问题,用系统的整体性、结构性、层次性和开放性原理,分析企业管理系统,从而更准确地把握企业管理理论的精髓,更好地运作企业;从辨证思维上来讲,要求企业看事物要一分为二,不能只注意到其有利的一面,同时还要看到其不利的一面;现实的管理中,人们由于缺乏对管理本身特点的深刻认识,造成管理失误和资源浪费,深入剖析管理科学观的内涵使我们认识到,管理行为是对资源的配置行为,管理决策对资源配置的影响是不可逆的。
\par
因此,在管理实践中必须树立管理的科学观,更加重视管理决策,提高科学管理的系统思考能力。
\section{结语}
自然辩证法确实在本文提及的几个方面发挥着重要的作用,我们应当应学习自然辩证法,提高自己的哲学素养和人文素质,进一步树立辩证唯物主义的世界观,掌握和运用科学的思维工具去探索自然规律。
% ==========References==========
\renewcommand{\refname}{参考文献}
% enter your .bib file name in the parentheses in the following line.
\bibliography{ref}

\clearpage


\end{document}
