\documentclass[UTF8,a4paper]{ctexart}

% ==========Preamble==========

\usepackage{apacite}
\usepackage{fancyhdr}
\usepackage{geometry}
\usepackage[font=small,labelfont=bf,labelsep=quad,format=hang,textfont=it]{caption}
\usepackage{booktabs}
\usepackage{graphicx}
\usepackage{float}

\pagestyle{plain}
\CTEXsetup[format=\Large\bfseries]{section}
\bibliographystyle{apacite}

% ==========Title==========

\title{\bfseries 自然辩证法在建筑、教育、企业管理中的应用 } 
\author{谈昊\quad2020E8013282037\quad217}

% Your name in the first blank and your additional information in \thanks{}
\date{\today}
% delete \today if you don't want the date

% ==========Document==========

\begin{document}
\maketitle

% ==========Abstract==========

\begin{center}
\parbox{130mm}{\zihao{-5}{\bfseries 摘\quad 要:}
% Abstract here
% An example is as follows
自然辩证法是研究自然界和科学技术发展的一般规律以及人类认识和改造自然的一般方法的科学,基于系统观、辩证观、科学观对管理实践的理论深化,在社会发展中发挥了重要的作用。本文详细阐述了其在建筑、教育、企业管理方面的重要作用。
\par
\vspace{1mm}
{\bfseries 关键词:}自然辩证法\quad 建筑 \quad 教育 \quad 企业管理 \quad 协调}
\end{center}

% ==========Body==========

% Example article

\section{前言}
自然辩证法是关于自然界和自然科学发展的普遍规律的科学,是马克思主义哲学的分支学科,是由马克思和恩格斯共同创立的。自然辩证法的创立不是来自于思辩和猜测,而是在19世纪自然科学发展、哲学发展和社会发展的科学文化背景下,经过长期潜心研究而建立的一个完整的、相对独立的科学理论知识体系,是人类科学认识发展的必然结果。自然辩证法的内容是阐释马克思主义的自然哲学、科学哲学和技术哲学,是关于自然界和科学技术发展的一般规律及人类认识和改造自然的一般方法的科学,是系统化、理论化的自然观、科学技术观和科学技术研究的方法论。自然辩证法是一个开放性的科学理论知识体系,随着科学技术的不断发展,自然辩证法的内容也不断得到证实、丰富和发展。\cite{RN14}
\section{自然辩证法在建筑中的应用}
近年来,由于现代化环境中科学技术的空前迅猛发展,人类认识自然、改造自然的能力逐步提高,建筑设计应遵循的自然辩证法原则得到了有力支持。同时,随着科学技术的快速发展,人类可以通过更多的方式来实现生态建筑设计,优化建筑设计和人类生活环境的质量。自然辩证法研究的对象是人与自然的整体关系,研究的目的是如何合理解决人与自然的矛盾,建筑是人类生存和社会发展的空间载体,人、自然、社会的发展需求是建筑师关注的重点,在设计中遵循自然辩证法成为其基本设计理念。工业革命以来,科学技术推动了生产方式的进步,社会经济不断发展,生活水平不断提高,消费者的购买力越来越强,随之而来的是自然资源的迅速流失,人类的贪得无厌导致了环境的恶化和资源的匮乏,生态危机开始到来。 发展道路。随着城市化的不断发展,建筑业有着广阔的市场前景,同时,环境保护和可持续发展的理论也显得尤为重要,生态建筑设计逐渐兴起,人们渴望得到一个良好的居住环境,渴望与大自然和青山绿水为伍。
\par
\cite{RN11}提出应尽量利用一切可以利用的建筑物和建筑材料,以有助于大大减少资源的消耗,降低消耗量,同时也可以减少建筑垃圾的产生,以达到对自然生态的保护,在建筑的设计过程中,重视旧有建筑的维护更新和对原有建筑的重复建设;其次,充分考虑原有的自然生态条件的利用率,注重协调建筑和自然生态环境的关系,以防施工时发生水土流失和其他自然灾害,并保持水土,调节气候,降低污染;设计中应当结合地理位置、方向等,采用自然通风、天然采光的方法,降低建筑对能源的依赖。

\section{自然辩证法在教育中的应用}
"自然辩证法概论 "是自然科学、技术科学、社会科学和思维科学的交叉学科,它涉及自然科学、技术科学、社会科学和思维科学,具有综合学科的性质。自然辩证法课程的普遍性、综合性、哲学性和跨学科性的特点,使其在目前我国高等教育课程中具有重要地位和作用。它弥补了我国高等教育课程中整体性课程和交叉学科的不足,对拓展学生的知识面,提高学生的综合思维能力具有重要作用。
\par
\cite{RN13}提出,与具体的各门自然科学相比,由于自然辩证法具有世界观和方法论的高度,使它有突出的德育功能。自然辩证法还具有将道德教育嵌套在智力教育中的特点。在自然辩证法课程中,对学生的思想教育渗透到科学技术专业中。它使学生通过对自然界和自然科学辩证发展规律的认识,以及对自然科学一般研究方法的理解和掌握,逐步建立起正确的自然界和自然科学观,这是智力教育的一部分。

\section{自然辩证法在企业管理中的应用}
企业管理是协调人与自然关系的实践活动,必须遵循自然辨证法。系统观、辨证观和科学观构成了自然辨证法的理论体系,指导企业管理的实践。
\par
\cite{RN12}提出从系统观赏来讲,管理者把企业看成一个系统,针对企业管理中经常遇到的问题,运用系统性、结构性、层次性和开放性的原则,分析企业管理制度,从而更准确地把握企业管理理论的精髓,更好地经营企业;从辨证思维看,要求企业一分为二地看待事物,不仅要注意到它有利的一面,还要看到它 在辨证思维方面,要求企业一分为二地看待事物,不仅要注意到它有利的一面,还要看到它不利的一面。在现实管理中,人们对管理本身的特点缺乏深刻的认识,造成了管理的失误和资源的浪费,深入分析科学管理观的内涵使我们认识到,管理行为是对资源的配置,管理决策对资源配置的影响是不可逆转的。
\par
因此,在管理实践中必须树立管理的科学观,更加重视管理决策,提高科学管理的系统思考能力。
\section{结语}
自然辩证法确实在本文提及的几个方面发挥着重要的作用,我们应当应学习自然辩证法,提高自己的哲学素养和人文素质,进一步树立辩证唯物主义的世界观,掌握和运用科学的思维工具去探索自然规律。
% ==========References==========
\renewcommand{\refname}{参考文献}
% enter your .bib file name in the parentheses in the following line.
\bibliography{ref}

\clearpage


\end{document}
