\documentclass[UTF8,a4paper]{ctexart}

% ==========Preamble==========
\usepackage{graphicx}
\usepackage{apacite}
\usepackage{url}
\usepackage{fancyhdr}
\usepackage{geometry}
\usepackage[font=small,labelfont=bf,labelsep=quad,format=hang,textfont=it]{caption}
\usepackage{booktabs}
\usepackage{graphicx}
\usepackage{float}

\pagestyle{plain}
\CTEXsetup[format=\Large\bfseries]{section}
\bibliographystyle{apacite}

% ==========Title==========

\title{\bfseries CRISPR技术的伦理讨论 } 
\author{\bfseries 谈昊\quad2020E8013282037}

% Your name in the first blank and your additional information in \thanks{}
\date{}
% delete \today if you don't want the date

% ==========Document==========

\begin{document}
\maketitle

% ==========Body==========

\section{背景介绍}
\subsection{技术介绍}
靶向核酸酶是用来高精度改变介导基因组的有力工具。通过对基因进行剪切,实现基因表达的改变。CRISPR(常间回文重复序列丛集),可以看作细菌后天获得的的保护机制,类似于人体的特异性免疫。当在CRISPR中出现过的基因序列再次出现时,就会进行特异性识别,通过引导RNA,使其靠近Cas9蛋白(CRISPR相关蛋白),实现基因剪切。CRISPR技术可以看作是一种精准的基因“剪刀”,实现目标基因的编辑——删除、添加、激活和抑制。
\subsection{“基因编辑婴儿”事件始末}
基因编辑婴儿案是中国南方科技大学生物系副教授贺建奎和他的团队在2018年通过基因编辑技术修改了一对双胞胎婴儿胚胎细胞中的CCR5基因,试图让婴儿对艾滋病毒产生部分免疫力。理论上讲,经过基因编辑后的婴儿不会再受到艾滋病病毒的威胁。事件开始是,贺某在Youtube上传了自己关于基因编辑婴儿出生的相关视频。随后,两位人民网记者报道了这一事件,附有一定的正面评价,引起了轩然大波,令整个科学界以及各方人士震惊。之后不久,百位科学家联合署名,谴责贺某这一严重违反人类伦理的不道德科学行为。至此,这件事情被定性为及其不当的科研活动,再无反转。而基因治疗这件事情也再一次被推上了风口浪尖。最终,贺某及其伙伴被法院判处三年有期徒刑,处300万罚金。

\section{伦理是否阻碍了科学的发展}
在查阅相关资料的过程中,发现了一些有趣的言论。有网友说,“一群虫子因为所谓的理论畏缩不前,那伦理就与邪教无异”。这句话让我感到一阵寒意。从这位网友的角度来看,科学理应蓬勃发展,不应受到伦理以及世俗的限制。埃隆·马斯克创办SpaceX的目的,就是不满于当前科学发展的诸多限制,希望尽快实现星际旅行。从这个角度来讲,以一己之力来推动整个科学的发展,此处并不是指科学技术上的发展,而是主动打破科学上的某些限制、踏入科学的禁区,好像是行得通的。但是,若真如此,这个世界会变成什么样子呢?
\par
且不说,基因编辑是否会带来尚未查明的严重后果,例如,贺某事件中的婴儿是否会表露出其他严重的症状——“寿命缩短”等。单从社会伦理角度来讲,如果未来真的可以通过基因编辑技术来修复或者增强人类后代,那么是不是每个人都会成为自己理想中的“超人”呢?人类可能会在水底呼吸,或者长出像鸟一样的翅膀,更大胆一点,是不是会像植物一样进行光合作用,不需要额外的摄取食物。这一切看似很美好,但是隐藏着巨大的社会问题。当基因编辑变成潮流,类似于今天的“医美”——每个人都想变成最漂亮的自己,刻意地追求完美。当完美的标准趋于一致,人类的特异性会不会就此消失呢?每个人都是复制品,没有个性可言。
\par
基因编辑,目前尚无法保证全部的积极效应。也就是说,无法保证经过基因编辑的婴儿不会有其他的副作用,诸如,寿命缩短或者因脱靶效应带来的其他意想不到的问题。再者,当基因编辑婴儿长大后,面临繁殖问题,其本身基因进入人类基因库时,是否会造成编辑基因延续传递的问题。那么,这个基因影响的人群会越来越广泛,由于大样本效应,出现不可控“并发症”的机率会大大增加。如果打开了,基因编辑这扇大门,就好像打开了潘多拉魔盒,会引发更多的伦理问题。今天对基因编辑的许可是否会导致后续的克隆人的出现,再到出现人类无法克制的新类人物种。这一系列问题,在没有合适的答案之前,基因编辑的禁区还是不进为妙。


\section{启示}
贺某作为中国南方科技大学的副教授,身居高位,却以身试法,私自进行人体实验,违反了法律和社会伦理体系,理应受到科学界的唾弃和法律的制裁。在科研工作中,我们应当时时刻刻严守法律底线,不做违反伦理道德的科学实验。我的研究方向——联邦学习,牵扯到用户端的隐私保护,如何在保护用户隐私的情况下,合理使用用户的数据,就是我们现在探究的主要问题。在之前的一些案例中,也出现了私自使用用户数据进行大数据分析以及私人推荐的情况,在这方面国家的法律还比较单薄,企业的道德标准也不尽相同,所以监管起来比较困难。在2021年元旦正式实施的民法典中,就对这种现象做出了明确的规定。伦理问题正逐渐被大众所重视,在重视经济高质量发展的现阶段,如何让社会大众重新审视并建立起新的道德标准迫在眉睫。



\clearpage


\end{document}
